% Document and formatting
\documentclass[a4paper]{article}
\usepackage{fullpage}
\renewcommand{\baselinestretch}{1.15} 
\usepackage{booktabs}
\newcommand{\ra}[1]{\renewcommand{\arraystretch}{#1}}

% Math symbols
\usepackage{amsmath}
\usepackage{amsfonts}
\usepackage{mathtools}
\usepackage{amssymb}
\usepackage{amsthm}
\usepackage{bbm}
\usepackage{url}

% Colors and diagrams
\usepackage[dvipsnames]{xcolor}
\usepackage{tikz}

% Figures and references
\usepackage{caption}
\usepackage{paralist}


% Symbol commands
\renewcommand\epsilon{\varepsilon}
\renewcommand\subset{\subseteq}
%\renewcommand\phi{\varphi}
\newcommand\Z{\mathbb{Z}}
\newcommand\N{\mathbb{N}}
\newcommand\Q{\mathbb{Q}}
\newcommand\todo[1]{{\color{red}todo: #1}}
\newcommand\R{\mathbb{R}}
\newcommand\C{\mathbb{C}}
\newcommand\F{\mathcal{F}}
\renewcommand\P{\mathbb{P}}
\newcommand\E{\mathbb{E}}
\renewcommand{\phi}{\varphi}
\usepackage{tcolorbox}
\newtcbox{\mybox}{nobeforeafter,colframe=black!50,colback=white,boxrule=1.8pt,arc=5pt,
  boxsep=0pt,left=6pt,right=6pt,top=6pt,bottom=6pt,tcbox raise base}




% Operator commands
\newcommand{\given}{\, \vert \,}
\newcommand{\st}{\, : \,}



% Bibliography
\usepackage{natbib}
\bibliographystyle{abbrvnat}

\newcommand\Rune[1]{{\color{blue}Rune: #1}}
\newcommand\Jonas[1]{{\color{red}Jonas: #1}}
\newcommand\Niklas[1]{{\color{Fuchsia}Niklas: #1}}
\newcommand\Martin[1]{{\color{PineGreen}Nikolaj: #1}}

\title{COVID-19: Estimating infections from deaths rates \\ \# WirVsVirus\\
Group: CausalityVsCorona}
%\author{Rune Christiansen\thanks{krunechristiansen@math.ku.dk} \ and Jonas Peters\thanks{jonas.peters@math.ku.dk}}
\author{Rune Christiansen\thanks{krunechristiansen@math.ku.dk}, Phillip Mogensen\thanks{pbm@math.ku.dk}, Jonas Peters\thanks{jonas.peters@math.ku.dk}, Niklas Pfister\thanks{np@math.ku.dk}, Nikolaj Thams\thanks{thams@math.ku.dk}}
\date{\today}
\begin{document}
\maketitle


\begin{center}
\mybox{
\begin{minipage}{0.95\textwidth}
\centering
Access to accurate numbers of infections during an epidemic is
important to create useful
public policy
interventions
and evaluate their effect.
Due to various reasons, however, 
the confirmed cases in a country are 
believed to underestimate the true number.
The number of confirmed fatalities is often believed to be more reliable that the number of confirmed infections
and contains information about the total number of infected people,
too.
In this project, we estimate the total number of 
infected people aged 30 or above, from fatality data and 
age distributions.

\end{minipage}
}
\end{center}



\section{Why are there more COVID cases 
than the reported confirmed cases?}
There are various ways to test whether a patient is infected by the
COVID virus. E.g., it is possible to detect the virus from respiratory
samples.  Even if we assume perfect tests (no false positives and no
false negatives), the number of confirmed cases is less than the true
number of infections because not all infected cases get tested.
Furthermore, the countries have different testing policies concerning
whom gets tested and these may even change over time.  According to
\url{https://en.wikipedia.org/wiki/COVID-19_testing} (20.03.2020,
3:27pm), the number of tests per 1,000,000 people differs between 9
(Indonesia) and 26,865 (Iceland).  The difference between number of
deaths per 1,000 confirmed cases (e.g., Germany: 3.8, UK: 46.1;
20.3.2020, 16:09pm) indicates further differences in testing policies.
Also the definition of `confirmed case' changes between countries and
time (cf.\ China's change of policy in February).
% \todo{show figures and examples}
It is widely accepted that the officially confirmed cases
underestimate the number of total cases of infections.


\section{The idea}
We propose an to estimate the total number of infections using the 
number of COVID fatalities. The latter number is more reliable in 
that it is unlikely that many cases are missed. To do so, 
we require knowledge of the following numbers (measured at a certain point in time):
(1) the number of deaths in a certain age group $a$,
(2) the death rate given that a person belongs to age group $a$.
We can then estimate the total number of infected people 
(this differs from the active cases),
by dividing the number of deaths in age group $a$ by the death rate for that age group. (Clearly, this approach fails if the death rate in a certain age group equals zero. We discuss this point in Section~\ref{sec:zerodeathrate}.)
The death rates in (2) may be considered as parameters.
Their values may be provided by
background knowledge. 
In the app
these can parameters can be set by hand,
As standard values, we use the death rates 
measured in South Korea, 
where, supposedly, many people have been tested for COVID,
see
\url{https://en.wikipedia.org/wiki/Coronavirus_disease_2019#Prognosis}, 22.03.2020, 9:24pm.
%\todo{write about the idea to use fatalities per age group and
%death rates from South Korea (nothing yet about what happens without age group data, that comes below)}
%\todo{cite some other sources that are based on a similar idea.}
This idea relies on the number of deaths per age group. 
For many countries, however, 
these data are unavailable. Under some
assumptions, it is still possible to estimate the total number of infections, see Section~\ref{sec:noage}.
We do not claim that our idea is novel, see Section~\ref{sec:disclaimer}.



\section{Age groups with zero death rate} \label{sec:zerodeathrate}
In some age groups the death rate is estimated to be zero.
(In South Korea, this is the case for people under 30,
see \url{https://en.wikipedia.org/wiki/Coronavirus_disease_2019#Prognosis}, 22.3.2020, 7:18pm.)
If the death rate is zero, it is impossible 
to estimate the number of infected people by the number of deaths.
{\bf In this project, we there restrict ourselves to estimate the number
of infected people aged 30 or above.}


\section{Missing age group information} \label{sec:noage}
Currently, we cannot use the above idea to 
estimate the number of total cases in 
several countries. The reason is that 
while we found reported number of total fatalities 
for almost all countries 
(see, e.g., the website {\footnotesize
\url{https://gisanddata.maps.arcgis.com/apps/opsdashboard/index.html#/bda7594740fd40299423467b48e9ecf6}},
last accessed on 22.03.2020, 7:30pm,
which is run by the Center for Systems Science and Engineering ({CSSE}) at Johns Hopkins University ({JHU}).
But for most of these countries we have not found the number of fatalities per age group. 
We believe that it would be highly informative to add these numbers.\footnote{The numbers 
are informative for two reasons: (1) It
becomes easier to estimate the 
total number of infected cases in the 
corresponding country.
(2) They help us to better understand Assumption A2. Having these data from 
more countries, for example, 
allows us to 
provide more accurate uncertainty bounds for 
the remaining countries, see `Uncertainties of Assumption A1 and A2'.}
We therefore need to estimate the number of fatalities per age group using data available from the other countries.


\section{... and a possible solution}
asd \todo{Describe the main idea}
Below, under `Modeling Framework', we describe the idea in more detail. 
We hope that it states
clearly which assumptions the estimator
relies on.





\section{Disclaimer} \label{sec:disclaimer}
\todo{
put 
\citet{linton2020epidemiological}
at the right place}
We 
developed this idea, implemented it, and 
wrote the 
document in a short amount of time. Please tell us if you find any typos 
in the document or possible errors in 
the calculations. 
Also, we are certain that there is a lot of 
related work to our approach that is not properly cited. 
Thus, we do not claim that our idea is original, we just wanted
to implement sth this weekend.
If you know related relevant work, please tell us.
You can find our email addresses above.
Some of the main assumptions underlying our prediction are described 
as Assumptions~A1 and~A2 in Section~\ref{sec:model} and
Section~\ref{sec:uncert} describes possible reasons for further uncertainty.


\section{Modeling framework} \label{sec:model}
Consider the following variables of an individual in country $c$: 
\begin{compactitem}
\item $A \in \N$ denotes age (we assume age to be constant over $t$)
\item $I \in \{0,1\}$ infection indicator
\item $D \in \{0,1\}$ the indicator for a Corona-related death 
\item $\tau > 0$ the time from infection to possible death
\item $\P_c(D_{t+\tau} = 1 \given I_t = 1, A_t = a)$ is the probability of dying, given one is infected and in age group $a$.
\end{compactitem}



For each country, we are given data from several individuals and time points:

\begin{itemize}
%\item $N_c$: number of individuals in country $c$
%\item $A_{c,i}$: age of individual $i$ in country $c$
%\item $I_{c,i,t}$: binary indicator of infection for individual $i$ in country $c$ at time $t$
%\item $D_{c,i,t}$: binary indicator of Corona-related death for individual $i$ in country $c$ at time $t$
\item $Y_{c,t}(a)$: number of infected individuals in country $c$ at age $a$ at time $t$
\item $X_{c,t}(a)$: number of Corona-related deaths in country $c$ at age $a$ at time $t$
\item $Y_{c,t} = \sum_a Y_{c,t}(a)$: total number of infected individuals in country $c$ at time $t$
\item $X_{c,t} = \sum_a X_{c,t}(a)$: total number of Corona-related deaths in country $c$ at time $t$
%\item $\tau$: time from infection to possible death
%\item $p_c(a) = \P(A_{c,i} = a)$: marginal age distribution of country $c$
%\item $p_{I}(a) = \P(I_{c,i,t} = 1 \given A_{c,i} = a)$: infection rate at age $a$
%\item $p_{D}(a) = \P(D_{c,i,t+\tau} = 1 \given I_{c,i,t} = 1, A_{c,i} = a)$ death rate at age $a$
\end{itemize}
%Both $p_I$ and $p_D$ are assumed to be the same across all countries $c$ and all time points $t$. 
%
Our goal is to estimate $Y_{c,t-\tau}$ from $X_{c,t}(a)$ (if available) or from $X_{c,t}$. 





Clearly, $\P_c(I_{t} = 1 \given A = a)$
differs between countries and changes throughout time.

\paragraph{Assumption A1} Let $a$ be fixed. Then, for all $c$, we have
\begin{equation}
P_c(D_{t+\tau} = 1\,|\,I_t = 1, A = a)
\end{equation}
are the same. 
Currently, we use the data from South Korea to 
estimate these numbers (see also `Uncertainties regarding A1 and A2' below). 



\paragraph{Assumption A2}
Let $\tilde{a}$ be fixed. Then, for all $c$,
\begin{equation} \label{eq:a2equiv}
\frac{P_c(D_t = 1\,|\,A = \tilde{a})}
{\sum_a \P_c(D_t = 1\,|\,A = a)}
\end{equation}
is the same.





\paragraph{Equivalent formulation of A2}
Assumption A2 is equivalent to the following assumption. 
Let $a$ and $\tilde{a}$ be fixed. Then, for all $c$, the ratios 
\begin{equation} \label{eq:a2equiv2}
\frac{P_c(D_t = 1\,|\,A = a)}{P_c(D_t = 1\,|\,A = \tilde{a})}
\end{equation}
are the same. 

{\color{gray} This can be seen as follows. Assume first that 
that Equation~\eqref{eq:a2equiv2} holds. Let $\tilde{a}$ be fixed. We then have 
\begin{align*}
\frac{P_c(D_t = 1\,|\,A = \tilde{a})}
{\sum_a \P_c(D_t = 1\,|\,A = a)}
&= 
\frac{1}{\frac{\sum_a \P_c(D_t = 1\,|\,A = a)}{P_c(D_t = 1\,|\,A = \tilde{a})}
}\\
&= 
\frac{1}{\sum_a \frac{P_c(D_t = 1\,|\,A = a)}{P_c(D_t = 1\,|\,A = \tilde{a})}
}
\end{align*}
Now, assume 
A2 holds. Then, Equation~\eqref{eq:a2equiv2} holds because of the following argument:  Let $\tilde{a}$ and $a^*$ be fixed. We then have
\begin{align*}
\frac{P_c(D_t = 1\,|\,A = a^*)}
{P_c(D_t = 1\,|\,A = \tilde{a})}
&= 
\frac{P_c(D_t = 1\,|\,A = a^*)/\sum_a \P_c(D_t = 1\,|\,A = a)
}
{P_c(D_t = 1\,|\,A = \tilde{a})/\sum_a \P_c(D_t = 1\,|\,A = a)
}
%&=\frac{P_{\tilde{c}}(D_t = 1\,|\,A = a)}
%{P_{\tilde{c}}(D_t = 1\,|\,A = \tilde{a})}.
\end{align*}}


\paragraph{Equivalent formulation of A2}
Assumption A2 is also equivalent to the following assumption. 
Let $a$ and $\tilde{a}$ be fixed. Then, for all $c$
\begin{equation}
\frac{P_c(I_t = 1\,|\,A = a)}{P_c(I_t = 1\,|\,A = \tilde{a})} 
\end{equation}
is invariant.
{\color{gray} This can be seen as follows.
\begin{align*}
\frac{P_c(I_t = 1\,|\,A = a)}{P_c(I_t = 1\,|\,A = \tilde{a})} 
&= k_1
\frac{P_c(I_t = 1,A = a)P_c(A = \tilde{a})
P(D_t = 1\,|\,I_t = 1, A = a)}
{P_c(I_t = 1,A = \tilde{a})P_c(A = a)P(D_t = 1\,|\,I_t = 1, A = \tilde{a})} \\
&= k_1
\frac{P_c(A = \tilde{a})
P_c(D_t = 1,I_t = 1, A = a)}
{P_c(A = a)P_c(D_t = 1,I_t = 1, A = \tilde{a})}\\
&= k_1
\frac{P_c(A = \tilde{a})
P_c(D_t = 1,A = a)}
{P_c(A = a)P_c(D_t = 1, A = \tilde{a})}\\
&= k_1
\frac{P_c(D_t = 1\,|\,A = a)}
{P_c(D_t = 1\,|\, A = \tilde{a})}.
\end{align*}
}



\paragraph{Motivation for A2}
The infection rate $a \mapsto \P_c(I_{t} = 1 \given A_t = a)$ within 
each country changes through time with the same factor for each age. If, for each country, there exists a function $g_c$, such that for every $t$, the joint
density over $(A_t,I_t,D_{t+\tau})$ is given as
%
\begin{align*}
\P_c(A_t = a, D_{t+\tau} = 1, I_t = 1)	&= \P_c(D_{t+\tau} = 1 \given I_t = 1, A_t = a)  \P_c(I_t = 1 \given A_t = a) \P_c(A_t = a) \\
															&= \underbrace{\P(D_{\tau} = 1 \given I_0 = 1, A_0 = a)}_{=:p_D(a)} g_c(t) \underbrace{\P(I_1 = 1 \given A_1 = a)}_{=:p_I(a)} \underbrace{\P_c(A_t = a)}_{=:p_c(a)} \\
															&=  g_c(t) p_D(a) p_I(a) p_c(a)
%															&= \underbrace{\P(D_{\tau} = 1 \given I_0 = 1, A_0 = a)}_{=:p_D(a)} \underbrace{\P(I_0 = 1 \given A_0 = a)}_{:=p_I(a)} \underbrace{\P_c(A_0 = a)}_{:=p_c(a)}.
\end{align*}
%
it follows that 
$$
\frac{\P_c(I_t = 1 \given A_t = a)}{\P_c(I_t = 1 \given A_t = \tilde{a})} 
= 
\frac{
g_c(t) \P(I_1 = 1 \given A_1 = a)}{
g_c(t) \P(I_1 = 1 \given A_1 = \tilde{a})}
$$
is the same for all $c$. 


\paragraph{Uncertainties regarding A1 and A2}
asd
\todo{write sth regarding}


\paragraph{Code}
In the code, we currently compute the following. 
For one country $c$, e.g. South Korea, we compute, for all $a$,
$$
\frac{n_c \P_c(D_t = 1\,|\,A = a)}
{\sum_a n_c \P_c(D_t = 1\,|\,A = a)}
=
\frac{P_c(D_t = 1\,|\,A = a)}
{\sum_a \P_c(D_t = 1\,|\,A = a)}
$$
and use this to estimate the total number 
of infected people in other countries.



%\paragraph{The estimator}
%JONAS IS WRITING HERE.

\paragraph{The estimator for known number of deaths per age group}
%
For every $a$ we have that 
$$X_{c,t}(a) \given Y_{c,t-\tau}(a) \sim \text{Binom}(Y_{c,t-\tau}(a), p_D(a)),$$
and we thus obtain estimates 
$$\hat{Y}_{c,t-\tau} = \sum_a \hat{Y}_{c,t-\tau}(a) =\sum_a
X_{c,t}(a) / p_D(a)$$
Given that we only have access to $X_{c,t}(a)$ at some fixed time
$t^*$ we estimate it by scaling to the total deaths as follows
\begin{equation*}
  \hat{X}_{c,t}(a)=X_{c,t}\cdot\frac{X_{c,t^*}(a)}{\sum_a X_{c,t^*}(a)}
\end{equation*}
We can also compute confidence bounds using the Binomial distribution. This
leads to the bounds
\begin{equation*}
  \hat{Y}_{c,t-\tau}^{\text{lower}} = \inf\{n\in\N \,\vert\, \P(X\geq
  x)>\frac{\alpha}{2}\}
  \quad\text{and}\quad
  \hat{Y}_{c,t-\tau}^{\text{upper}} = \sup\{n\in\N \,\vert\, \P(X\leq
  x)>\frac{\alpha}{2}\}.
\end{equation*}

\paragraph{The estimator for unknown number of deaths per age group}
%
If the values $X_{c,t}(a)$ are unobserved, we can estimate these as follows. By definition of $X_{c,t}$ we get that
$$X_{c,t}(a) \given X_{c,t} \sim \text{Binom}(X_{c,t}, \P_c(A_{t-\tau} = a \given D_{t} = 1))$$
and hence obtain that
\begin{align*}
\hat{X}_{c,t}(a) = \E[X_{c,t}(a) \given X_{c,t}] 	
&= X_{c,t} \cdot \P_c(A_{t-\tau} = a \given D_{t} = 1) \\
     &= 
X_{c,t} \cdot 
\frac{  
    \P_c(D_t = 1, A = a)}
  {\P_c(D_t = 1)}\\
    &= 
X_{c,t} \cdot 
\frac{  
    \P_c(D_t = 1\given A = a)
     \cdot \P_c(A = a)}
  {\sum_{a^*}  
    \P_c(D_t = 1\given A = a^*) \cdot \P_c(A = a^*)
    }\\
&= 
X_{c,t} \cdot 
\frac{
  \frac{
    \P_c(D_t = 1\given A = a)}
    {\sum_{\tilde{a}} \P_c(D_t = 1 \given A = \tilde{a})} \cdot \P_c(A = a)}
  {\sum_{a^*}  
  \frac{
    \P_c(D_t = 1\given A = a^*)}
    {\sum_{\tilde{a}} \P_c(D_t = 1 \given A = \tilde{a})} \cdot \P_c(A = a^*)}
\end{align*}

\Niklas{todo: Get bounds here.}


\section{The app}
The app 
can be found at 
\url{http://shiny.science.ku.dk/pbm/COVID19/}.


\section{Further wources of uncertainty} \label{sec:uncert}
There is one more source of uncertainty. 
Suppose that in an age group, the death rate is $0.1$ and we have 
$0$ fatalities in that age group.
A point estimate might then say that there are $0$ infected persons
in that age group. However, having $300$ infected persons, say,
is a reasonable explanation of the data, too (in that case, we would expect $0.1\cdot 300 = 0.3$ fatalities).
Currently, our estimate does not include this source of uncertainty. 
\todo{We are not accounting for that yet, Niklas and Nikolaj, correct?}
Furthermore, one should take into account the uncertainty of 
death rates estimates. 


\section{Mathematics and real life}
Assumption~A1 looks like an assumption about mathematics. 
But it is not. 
The conditional probability does not 
only describe how
the virus affects humans, 
but also how the health system treats the patients. 
There is a lot of staff working hard to 
keep this probability as small as possible. 
Thank, all of you working in the health systems in all different countries, for 
your efforts to keep this number small. 




\section{Extrapolating infections into present/future}
We would like to work on this. but we are running out of time for this weekend, so we may come back to that question only later.

\section{Conclusions}

\begin{center}
\mybox{
\begin{minipage}{0.96\textwidth}
\centering
Our analysis suggests that the true number of infected people is a lot higher than the reported numbers -- please respect social distancing to avoid overburdening the health system. And stay safe!
\end{minipage}
}
\end{center}

\bibliography{ref}

\Jonas{old stuff

\section{Estimating past infections}

Let $A \in \N$ denote age, $I \in \{0,1\}$ infection indicator, $D \in \{0,1\}$ the indicator for a Corona-related death and $\tau > 0$
the time from infection to possible death. We assume that the death rate $a \mapsto \P_c(D_{t+\tau} = 1 \given I_t = 1, A_t = a)$ is 
the same for all countries $c$ and all time points $t$, and that the infection rate $a \mapsto \P_c(I_{t} = 1 \given A_t = a)$ within 
each country changes through time with the same factor for each age. For each country, there exists a function $g_c$, such that for every $t$, the joint
density over $(A_t,I_t,D_{t+\tau})$ is given as
%
\begin{align*}
\P_c(A_t = a, D_{t+\tau} = 1, I_t = 1)	&= \P_c(D_{t+\tau} = 1 \given I_t = 1, A_t = a)  \P_c(I_t = 1 \given A_t = a) \P_c(A_t = a) \\
															&= \underbrace{\P(D_{\tau} = 1 \given I_0 = 1, A_0 = a)}_{=:p_D(a)} g_c(t) \underbrace{\P(I_1 = 1 \given A_1 = a)}_{=:p_I(a)} \underbrace{\P_c(A_t = a)}_{=:p_c(a)} \\
															&=  g_c(t) p_D(a) p_I(a) p_c(a)
%															&= \underbrace{\P(D_{\tau} = 1 \given I_0 = 1, A_0 = a)}_{=:p_D(a)} \underbrace{\P(I_0 = 1 \given A_0 = a)}_{:=p_I(a)} \underbrace{\P_c(A_0 = a)}_{:=p_c(a)}.
\end{align*}
%
It follows that 
$$
\frac{\P_c(I_t = 1 \given A_t = a)}{\P_c(I_t = 1)} = \frac{g_c(t) \P(I_1 = 1 \given A_1 = a)}{g_c(t) \P(I_1 = 1)} = \frac{\P(I_1 = 1 \given A_1 = a)}{ \P(I_1 = 1)}
$$
is the same for all $c$ and for all $t$. 


For each country, we are given data from several individuals and time points:

\begin{itemize}
%\item $N_c$: number of individuals in country $c$
%\item $A_{c,i}$: age of individual $i$ in country $c$
%\item $I_{c,i,t}$: binary indicator of infection for individual $i$ in country $c$ at time $t$
%\item $D_{c,i,t}$: binary indicator of Corona-related death for individual $i$ in country $c$ at time $t$
\item $Y_{c,t}(a)$: number of infected individuals in country $c$ at age $a$ at time $t$
\item $X_{c,t}(a)$: number of Corona-related deaths in country $c$ at age $a$ at time $t$
\item $Y_{c,t} = \sum_a Y_{c,t}(a)$: total number of infected individuals in country $c$ at time $t$
\item $X_{c,t} = \sum_a X_{c,t}(a)$: total number of Corona-related deaths in country $c$ at time $t$
%\item $\tau$: time from infection to possible death
%\item $p_c(a) = \P(A_{c,i} = a)$: marginal age distribution of country $c$
%\item $p_{I}(a) = \P(I_{c,i,t} = 1 \given A_{c,i} = a)$: infection rate at age $a$
%\item $p_{D}(a) = \P(D_{c,i,t+\tau} = 1 \given I_{c,i,t} = 1, A_{c,i} = a)$ death rate at age $a$
\end{itemize}
%Both $p_I$ and $p_D$ are assumed to be the same across all countries $c$ and all time points $t$. 
%
Our goal is to estimate $Y_{c,t-\tau}$ from $X_{c,t}(a)$ (if available) or from $X_{c,t}$. 

\subsection{Known number of deaths per age group}
%
For every $a$ we have that 
$$X_{c,t}(a) \given Y_{c,t-\tau}(a) \sim \text{Binom}(Y_{c,t-\tau}(a), p_D(a)),$$
and we thus obtain estimates 
$$\hat{Y}_{c,t-\tau} = \sum_a \hat{Y}_{c,t-\tau}(a) =\sum_a  X_{c,t}(a) / p_D(a)$$

\subsection{Unknown number of deaths per age group}
%
If the values $X_{c,t}(a)$ are unobserved, we can estimate these as follows. By definition of $X_{c,t}$ we get that
$$X_{c,t}(a) \given X_{c,t} \sim \text{Binom}(X_{c,t}, \P_c(A_{t-\tau} = a \given D_{t} = 1))$$
and hence obtain that
\begin{align*}
\hat{X}_{c,t}(a) = \E[X_{c,t}(a) \given X_{c,t}] 	&= X_{c,t} \cdot \P_c(A_{t-\tau} = a \given D_{t} = 1) \\
																		&= X_{c,t} \cdot \frac{\P_c(D_t = 1 \given A_{t-\tau} = a) \P_c(A_{t-\tau} = a)}{\P_c(D_t = 1)} \\
																		&= X_{c,t} \cdot \frac{\P_c(D_t = 1, I_{t-\tau} = 1 \given A_{t-\tau} = a) \P_c(A_{t-\tau} = a)}{\P_c(D_t = 1)} \\
																		&= X_{c,t} \cdot \frac{\P_c(D_t = 1\given I_{t-\tau} = 1 A_{t-\tau} = a) \P_c(I_{t-\tau} = 1 \given A_{t-\tau} = a) \P_c(A_{t-\tau} = a)}{\P_c(D_t = 1)} \\
																		&= X_{c,t} \cdot \P_D(a) p_c(a) \frac{\P_c(I_{t-\tau} = 1 \given A_{t-\tau} = a) }{\P_c(D_t = 1)} \\
%																		&= X_{c,t} \cdot \frac{p_D(a) p_I(a) p_c(a)}{\P_c(D_\tau = 1)} \\
%																		&= X_{c,t} \cdot \frac{p_D(a) p_I(a) p_c(a)}{\P_c(D_\tau = 1, I_0 = 1)} \\
%																		&= X_{c,t} \cdot \frac{p_D(a) p_I(a) p_c(a)}{\sum_a p_D(a) p_I(a) p_c(a)} \\
\end{align*}
}

\end{document}

%%% Local Variables:
%%% mode: latex
%%% TeX-master: t
%%% End:
