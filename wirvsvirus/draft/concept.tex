% Document and formatting
\documentclass[a4paper]{article}
\usepackage{fullpage}
\renewcommand{\baselinestretch}{1.15} 
\usepackage{booktabs}
\newcommand{\ra}[1]{\renewcommand{\arraystretch}{#1}}

% Math symbols
\usepackage{amsmath}
\usepackage{amsfonts}
\usepackage{mathtools}
\usepackage{amssymb}
\usepackage{amsthm}
\usepackage{bbm}

% Colors and diagrams
\usepackage[dvipsnames]{xcolor}
\usepackage{tikz}

% Figures and references
\usepackage{caption}


% Symbol commands
\renewcommand\epsilon{\varepsilon}
\renewcommand\subset{\subseteq}
%\renewcommand\phi{\varphi}
\newcommand\Z{\mathbb{Z}}
\newcommand\N{\mathbb{N}}
\newcommand\Q{\mathbb{Q}}
\newcommand\R{\mathbb{R}}
\newcommand\C{\mathbb{C}}
\newcommand\F{\mathcal{F}}
\renewcommand\P{\mathbb{P}}
\newcommand\E{\mathbb{E}}
\renewcommand{\phi}{\varphi}




% Operator commands
\newcommand{\given}{\, \vert \,}
\newcommand{\st}{\, : \,}



% Bibliography
\usepackage{natbib}
\bibliographystyle{abbrvnat}

\newcommand\Rune[1]{{\color{blue}Rune: #1}}
\newcommand\Jonas[1]{{\color{red}Jonas: #1}}
\newcommand\Niklas[1]{{\color{Fuchsia}Niklas: #1}}
\newcommand\Martin[1]{{\color{PineGreen}Nikolaj: #1}}

\title{COVID-19: Estimating infections from deaths rates \\ \# WirVsVirus}
%\author{Rune Christiansen\thanks{krunechristiansen@math.ku.dk} \ and Jonas Peters\thanks{jonas.peters@math.ku.dk}}
\author{Jonas Peters, Niklas Pfister, Nikolaj Thams, Rune Christiansen}
\date{\today}

\begin{document}
\maketitle

\section{Estimating past infections}

Let $A \in \N$ denote age, $I \in \{0,1\}$ infection indicator, $D \in \{0,1\}$ the indicator for a Corona-related death and $\tau > 0$
the time from infection to possible death. We assume that the death rate $a \mapsto \P_c(D_{t+\tau} = 1 \given I_t = 1, A_t = a)$ is 
the same for all countries $c$ and all time points $t$, and that the infection rate $a \mapsto \P_c(I_{t} = 1 \given A_t = a)$ within 
each country changes through time with the same factor for each age. For each country, there exists a function $g_c$, such that for every $t$, the joint
density over $(A_t,I_t,D_{t+\tau})$ is given as
%
\begin{align*}
\P_c(A_t = a, D_{t+\tau} = 1, I_t = 1)	&= \P_c(D_{t+\tau} = 1 \given I_t = 1, A_t = a)  \P_c(I_t = 1 \given A_t = a) \P_c(A_t = a) \\
															&= \underbrace{\P(D_{\tau} = 1 \given I_0 = 1, A_0 = a)}_{=:p_D(a)} g_c(t) \underbrace{\P(I_1 = 1 \given A_1 = a)}_{=:p_I(a)} \underbrace{\P_c(A_t = a)}_{=:p_c(a)} \\
															&=  g_c(t) p_D(a) p_I(a) p_c(a)
%															&= \underbrace{\P(D_{\tau} = 1 \given I_0 = 1, A_0 = a)}_{=:p_D(a)} \underbrace{\P(I_0 = 1 \given A_0 = a)}_{:=p_I(a)} \underbrace{\P_c(A_0 = a)}_{:=p_c(a)}.
\end{align*}
%
It follows that 
$$
\frac{\P_c(I_t = 1 \given A_t = a)}{\P_c(I_t = 1)} = \frac{g_c(t) \P(I_1 = 1 \given A_1 = a)}{g_c(t) \P(I_1 = 1)} = \frac{\P(I_1 = 1 \given A_1 = a)}{ \P(I_1 = 1)}
$$
is the same for all $c$ and for all $t$. 


For each country, we are given data from several individuals and time points:

\begin{itemize}
%\item $N_c$: number of individuals in country $c$
%\item $A_{c,i}$: age of individual $i$ in country $c$
%\item $I_{c,i,t}$: binary indicator of infection for individual $i$ in country $c$ at time $t$
%\item $D_{c,i,t}$: binary indicator of Corona-related death for individual $i$ in country $c$ at time $t$
\item $Y_{c,t}(a)$: number of infected individuals in country $c$ at age $a$ at time $t$
\item $X_{c,t}(a)$: number of Corona-related deaths in country $c$ at age $a$ at time $t$
\item $Y_{c,t} = \sum_a Y_{c,t}(a)$: total number of infected individuals in country $c$ at time $t$
\item $X_{c,t} = \sum_a X_{c,t}(a)$: total number of Corona-related deaths in country $c$ at time $t$
%\item $\tau$: time from infection to possible death
%\item $p_c(a) = \P(A_{c,i} = a)$: marginal age distribution of country $c$
%\item $p_{I}(a) = \P(I_{c,i,t} = 1 \given A_{c,i} = a)$: infection rate at age $a$
%\item $p_{D}(a) = \P(D_{c,i,t+\tau} = 1 \given I_{c,i,t} = 1, A_{c,i} = a)$ death rate at age $a$
\end{itemize}
%Both $p_I$ and $p_D$ are assumed to be the same across all countries $c$ and all time points $t$. 
%
Our goal is to estimate $Y_{c,t-\tau}$ from $X_{c,t}(a)$ (if available) or from $X_{c,t}$. 

\subsection{Known number of deaths per age group}
%
For every $a$ we have that 
$$X_{c,t}(a) \given Y_{c,t-\tau}(a) \sim \text{Binom}(Y_{c,t-\tau}(a), p_D(a)),$$
and we thus obtain estimates 
$$\hat{Y}_{c,t-\tau} = \sum_a \hat{Y}_{c,t-\tau}(a) =\sum_a  X_{c,t}(a) / p_D(a)$$

\subsection{Unknown number of deaths per age group}
%
If the values $X_{c,t}(a)$ are unobserved, we can estimate these as follows. By definition of $X_{c,t}$ we get that
$$X_{c,t}(a) \given X_{c,t} \sim \text{Binom}(X_{c,t}, \P_c(A_{t-\tau} = a \given D_{t} = 1))$$
and hence obtain that
\begin{align*}
\hat{X}_{c,t}(a) = \E[X_{c,t}(a) \given X_{c,t}] 	&= X_{c,t} \cdot \P_c(A_{t-\tau} = a \given D_{t} = 1) \\
																		&= X_{c,t} \cdot \frac{\P_c(D_t = 1 \given A_{t-\tau} = a) \P_c(A_{t-\tau} = a)}{\P_c(D_t = 1)} \\
																		&= X_{c,t} \cdot \frac{\P_c(D_t = 1, I_{t-\tau} = 1 \given A_{t-\tau} = a) \P_c(A_{t-\tau} = a)}{\P_c(D_t = 1)} \\
																		&= X_{c,t} \cdot \frac{\P_c(D_t = 1\given I_{t-\tau} = 1 A_{t-\tau} = a) \P_c(I_{t-\tau} = 1 \given A_{t-\tau} = a) \P_c(A_{t-\tau} = a)}{\P_c(D_t = 1)} \\
																		&= X_{c,t} \cdot \P_D(a) p_c(a) \frac{\P_c(I_{t-\tau} = 1 \given A_{t-\tau} = a) }{\P_c(D_t = 1)} \\
%																		&= X_{c,t} \cdot \frac{p_D(a) p_I(a) p_c(a)}{\P_c(D_\tau = 1)} \\
%																		&= X_{c,t} \cdot \frac{p_D(a) p_I(a) p_c(a)}{\P_c(D_\tau = 1, I_0 = 1)} \\
%																		&= X_{c,t} \cdot \frac{p_D(a) p_I(a) p_c(a)}{\sum_a p_D(a) p_I(a) p_c(a)} \\
\end{align*}

\section{Extrapolating infections into present/future}

\end{document}%