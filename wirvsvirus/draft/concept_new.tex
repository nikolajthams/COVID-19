% Document and formatting
\documentclass[a4paper]{article}
\usepackage{fullpage}
\renewcommand{\baselinestretch}{1.15} 
\usepackage{booktabs}
\newcommand{\ra}[1]{\renewcommand{\arraystretch}{#1}}

% Math symbols
\usepackage{amsmath}
\usepackage{amsfonts}
\usepackage{mathtools}
\usepackage{amssymb}
\usepackage{amsthm}
\usepackage{bbm}
\usepackage{hyperref}
\usepackage{url}

\usepackage{fancyvrb}

% Colors and diagrams
\usepackage[dvipsnames]{xcolor}
\usepackage{tikz}

% Figures and references
\usepackage{caption}
\usepackage{paralist}

%\theoremstyle{remark}
\newtheorem{ass}{Assumption}
\newtheorem{prop}{Proposition}


% Symbol commands
\renewcommand\epsilon{\varepsilon}
\renewcommand\subset{\subseteq}
%\renewcommand\phi{\varphi}
\newcommand\Z{\mathbb{Z}}
\newcommand\N{\mathbb{N}}
\newcommand\Q{\mathbb{Q}}
\newcommand\todo[1]{{\color{red}todo: #1}}
\newcommand\R{\mathbb{R}}
\newcommand\C{\mathbb{C}}
\newcommand\F{\mathcal{F}}
\renewcommand\P{\mathbb{P}}
\newcommand\E{\mathbb{E}}
\renewcommand{\phi}{\varphi}
\usepackage{tcolorbox}
\newtcbox{\mybox}{nobeforeafter,colframe=black!50,colback=white,boxrule=1.8pt,arc=5pt,
  boxsep=0pt,left=6pt,right=6pt,top=6pt,bottom=6pt,tcbox raise base}




% Operator commands
\newcommand{\given}{\, \vert \,}
\newcommand{\st}{\, : \,}



% Bibliography
\usepackage{natbib}
\bibliographystyle{abbrvnat}

\newcommand\Rune[1]{{\color{blue}Rune: #1}}
\newcommand\Jonas[1]{{\color{red}Jonas: #1}}
\newcommand\Niklas[1]{{\color{Fuchsia}Niklas: #1}}
\newcommand\Martin[1]{{\color{PineGreen}Nikolaj: #1}}

\title{COVID-19: Estimating infections from deaths rates \\ \# WirVsVirus\\
Group: CausalityVsCorona}
%\author{Rune Christiansen\thanks{krunechristiansen@math.ku.dk} \ and Jonas Peters\thanks{jonas.peters@math.ku.dk}}
\author{Rune Christiansen\thanks{krunechristiansen@math.ku.dk}, Phillip Mogensen\thanks{pbm@math.ku.dk}, Jonas Peters\thanks{jonas.peters@math.ku.dk}, Niklas Pfister\thanks{np@math.ku.dk}, Nikolaj Thams\thanks{thams@math.ku.dk}}
\date{\today}
\begin{document}
\maketitle


\begin{center}
\mybox{
\begin{minipage}{0.95\textwidth}
  \centering Access to accurate numbers of infections during an
  epidemic is important to create useful public policy interventions
  and evaluate their effect.  Due to various reasons, however, the
  confirmed cases in a country are believed to underestimate the true
  number \citep{li2020substantial}. The number of confirmed fatalities
  is often believed to be more reliable than the number of confirmed
  infections and contains information about the total number of
  infected people, too.  In this project, we estimate the total number
  of infected people aged 30 or above, from fatality data and age
  distributions.
\end{minipage}
}
\end{center}



\section{Why are there more COVID cases 
than the reported confirmed cases?}
There are various ways to test whether a patient is infected by the
COVID virus. E.g., it is possible to detect the virus from respiratory
samples.  Even if we assume perfect tests (no false positives and no
false negatives), the number of confirmed cases is less than the true
number of infections because not all infected cases get tested.
Furthermore, the countries have different testing policies concerning
whom gets tested and these may even change over time.  According to
\url{https://en.wikipedia.org/wiki/COVID-19_testing} (20.03.2020,
3:27pm), the number of tests per 1,000,000 people differs between 9
(Indonesia) and 26,865 (Iceland).  The difference between number of
deaths per 1,000 confirmed cases (e.g., Germany: 3.8, UK: 46.1;
20.3.2020, 16:09pm) indicates further differences in testing policies.
Also the definition of `confirmed case' changes between countries and
time (cf.\ China's change of policy in February).
% \todo{show figures and examples}
It is widely accepted that the officially confirmed cases
underestimate the number of total cases of infections, see e.g.,
\cite{li2020substantial}.


\section{The idea}
We propose to estimate the total number of infections using the 
number of COVID fatalities. The latter number is more reliable in 
that it is unlikely that many cases are missed. To do so, 
we require knowledge of the following numbers (measured at a certain point in time):
(i) the number of deaths in a certain age group $a$,
(ii) the case fatality rate given that a person belongs to age group $a$.
We can then estimate the total number of infected people 
(this differs from the active cases),
by dividing the number of deaths in age group $a$ by the case fatality rate for that age group. 
(Clearly, this approach fails if the case fatality rate in a certain age group equals zero. We discuss this point in Section~\ref{sec:zerodeathrate}.)
Our method is readily implemented in \Verb+R+ and is available as ShinyApp
at \url{http://shiny.science.ku.dk/pbm/COVID19%20-%20Copy/}.

The case fatality rates in (ii) may be considered as parameters.
Their values may be provided by
background knowledge. 
In the app,
these parameters can be set by hand;
as standard values, we use the rates 
measured in South Korea, 
where, supposedly, many people have been tested for COVID,
see
\url{https://en.wikipedia.org/wiki/Coronavirus_disease_2019#Prognosis}, 22.03.2020, 9:24pm.
%\todo{write about the idea to use fatalities per age group and
%case fatality rates from South Korea (nothing yet about what happens without age group data, that comes below)}
%\todo{cite some other sources that are based on a similar idea.}
Ideally, the deaths per age group (i) can be directly calculated from the data. 
For many countries, however, this information is not available.
Under some assumptions, it is still possible to estimate the total number of infections
from the total number of deaths, see Section~\ref{sec:noage}.
We do not claim that our idea is novel, see Section~\ref{sec:disclaimer}.



\section{Age groups with zero case fatality rate} \label{sec:zerodeathrate}
In some age groups the case fatality rate is estimated to be zero.
(In South Korea, this is the case for people under 30,
see \url{https://en.wikipedia.org/wiki/Coronavirus_disease_2019#Prognosis}, 22.3.2020, 7:18pm.)
If the case fatality rate is zero, it is impossible 
to estimate the number of infected people by the number of deaths.
{\bf In this project, we therefore restrict ourselves to estimate the number
of infected people aged 30 or above.}


\section{Missing age group information} \label{sec:noage}
Currently, we cannot use the above idea to 
estimate the number of total cases in 
several countries. The reason is that 
while we found reported number of total fatalities 
for almost all countries 
(see, e.g., the website {\footnotesize
\url{https://gisanddata.maps.arcgis.com/apps/opsdashboard/index.html#/bda7594740fd40299423467b48e9ecf6}},
last accessed on 22.03.2020, 7:30pm,
which is run by the Center for Systems Science and Engineering ({CSSE}) at Johns Hopkins University ({JHU})), 
for many of these countries we have not found the number of fatalities per age group. 
We believe that it would be highly informative to add these numbers.\footnote{The numbers 
are informative for two reasons: (i) It
becomes easier to estimate the 
total number of infected cases in the 
corresponding country.
(ii) They help us to better understand Assumption~\ref{ass:2}. Having these data from 
more countries, for example, 
allows us to 
provide more accurate uncertainty bounds for 
the remaining countries, see `Uncertainties of Assumptions~\ref{ass:1}~and~\ref{ass:2}'.}
We therefore need to estimate the number of fatalities per age group using data available from the other countries.


\section{... and a possible solution}
Inferring the number of deaths per age group from the total number of deaths requires an additional assumption. 
Clearly, 
the probability of
being infected, given that one 
belongs to a certain age group, $50-59$, say, 
differs between different countries (e.g., in some countries, 
where the pandemic has started earlier, there may be many more cases, which results in generally higher probabilities).
Here, we assume that the \emph{relation} between these numbers for different age groups do not differ.
E.g., if 
in country $c$, the
probability of being infected, given one is in age group $50-59$,
is twice as large as the probability of being infected, given 
one belongs to the age group $60-69$, 
we assume that the same ratio, that is, 2, 
also appears when dividing the analogous probabilities in some other country $c^\prime \neq c$ --- even though the probabilities themselves may differ between the two countries. 
This assumption suffices to estimate the number of total infections
from the total number of deaths.
Below, in Section~\ref{sec:model}, `Modeling framework', we describe the idea in more detail, 
and also provide two equivalent formulations of the above assumption. 
%There, Assumption~\ref{ass:2} allows us to 
%estimate the total number of fatalities, even if 
%no age information for the number of deaths are available.
%In fact, above we have described an equivalent version of Assumption~\ref{ass:2} that is described in Equation~\eqref{eq:equiv3}.
%We hope that it states clearly which assumptions the estimator relies on.

%There is yet another way to look at this assumption.
%Again, the probabilities of 
%being infected, given that one belongs to a certain age group, $50-59$, say, differs between countries. If one assumes
%that this difference can be expressed by a multiplicative constant that is the same for all age groups, then 
%this implies Assumption~2, see Section~\ref{sec:model} below.




\section{Disclaimer} \label{sec:disclaimer}

We 
developed this idea, implemented it, and 
wrote the 
document in a short amount of time. Please tell us if you find any typos 
in the document or possible errors in 
the calculations. 
Also, we are certain that there is a lot of 
related work to our approach that is not properly cited. 
Thus, we do not claim that our idea is original, we just wanted
to implement it this weekend.
If you know related relevant work, please tell us.
You can find our email addresses above.
Some of the main assumptions underlying our prediction are described 
as Assumptions~\ref{ass:1}~and~\ref{ass:2} in Section~\ref{sec:model}, and
Section~\ref{sec:uncert} describes possible reasons for further uncertainty.


\section{Modeling framework} \label{sec:model}
The following variables describe an individual in country $c$ at time $t$:
\begin{compactitem}
\item $A \in \N$ denotes age (we assume age to be constant over time $t$)
\item $I_t \in \{0,1\}$ infection indicator, measured from illness onset
\item $D_t \in \{0,1\}$ the indicator for a Corona-related death
\end{compactitem}
Based on these variables, the case fatality rate for an 
individual from an age group $a$ in country $c$ is given as
\begin{equation*}
  P_c(D_{t+\tau} = 1 \given I_t = 1, A = a), 
\end{equation*}
where $\tau > 0$ is the time from illness onset to possible
death. This parameter is estimated to be 20.2 (95\% CI: 15.1, 29.5) days on
average according to \citet{jung2020real}. The slightly older paper
\cite{linton2020epidemiological} estimates this to
13.8 days (95\% CI: 11.8, 16.0).


%This parameter is the sum of the incubation time and the time 
%from illness onset to death, which are estimated to be 4.6 days (95\% CI: 3.3, 5.7) 
%and 13.8 days (95\% CI: 11.8, 16.0), respectively, according to \cite{linton2020epidemiological}. 
%The slightly more recent paper \citet{jung2020real} reports the time from onset to death 
%to be 20.2 (95\% CI: 15.1, 29.5). In our app, we take the most conservative estimate 
%$\tau = 4.6 + 20.2 \approx 25$ days as default value.


For each country, we observe (either one or both of) the following data
\begin{compactitem}
%\item $N_c$: number of individuals in country $c$
%\item $A_{c,i}$: age of individual $i$ in country $c$
%\item $I_{c,i,t}$: binary indicator of infection for individual $i$ in country $c$ at time $t$
%\item $D_{c,i,t}$: binary indicator of Corona-related death for individual $i$ in country $c$ at time $t$
\item $X_{c,t}(a)$: number of Corona-related deaths in country $c$ at
  time $t$ for age group $a$
\item $X_{c,t} = \sum_a X_{c,t}(a)$: total number of Corona-related
  deaths in country $c$ at time $t$
%\item $\tau$: time from infection to possible death
%\item $p_c(a) = \P(A_{c,i} = a)$: marginal age distribution of country $c$
%\item $p_{I}(a) = \P(I_{c,i,t} = 1 \given A_{c,i} = a)$: infection rate at age $a$
%\item $p_{D}(a) = \P(D_{c,i,t+\tau} = 1 \given I_{c,i,t} = 1, A_{c,i} = a)$ death rate at age $a$
\end{compactitem}
Our goal is to estimate the total number of infected
individuals in country $c$ at time $t$ given by
$$Y_{c,t} = \sum_a Y_{c,t}(a),$$ where $Y_{c,t}(a)$ is the number of
infected individuals in country $c$ at time $t$ for age group $a$.
For this we propose an approach to estimate $Y_{c,t-\tau}$ from
$X_{c,t}(a)$ if it is available and from $X_{c,t}$ otherwise. We require the following assumption.

%Clearly, $\P_c(I_{t} = 1 \given A = a)$
%differs between countries and changes throughout time.

\begin{ass}[Invariant case fatality rates] \label{ass:1}
For every fixed $a$, the probability 
\begin{equation*}
p_D(a) := P_c(D_{t+\tau} = 1\,|\,I_t = 1, A = a)
\end{equation*}
does not depend on $c$ nor $t$.
\end{ass}
%
Currently, we use the data from South Korea to 
estimate these numbers (see also Section~\ref{sec:uncert} below). 

\subsection{Known number of deaths per age group} \label{sec:known}
%
Consider a country for which the values $X_{c,t}(a)$ are observed. 
Under Assumption~\ref{ass:1}, we have that for every $a$, 
$$X_{c,t}(a) \given Y_{c,t-\tau}(a) \sim \text{Binom}(Y_{c,t-\tau}(a), p_D(a)),$$
and we thus obtain estimates
\begin{equation}
  \label{eq:Y1}
  \hat{Y}_{c,t-\tau} = \sum_a \hat{Y}_{c,t-\tau}(a) =\sum_a
  X_{c,t}(a) / \hat  p_D(a).
\end{equation}
If we only have access to the age-specific deaths at some fixed time
$t^*$, we estimate $X_{c,t}(a)$ at other time points by scaling to the total deaths as follows
\begin{equation*}
  \hat{X}_{c,t}(a)=X_{c,t}\cdot\frac{X_{c,t^*}(a)}{X_{c,t^*}}.
\end{equation*}
We can also compute confidence bounds for the estimator \eqref{eq:Y1}
using the Binomial distribution. This leads to the lower bound
\begin{equation*}
  \hat{Y}_{c,t-\tau}^{\text{lower}} = \sum_a \inf\{n\in\N \,\vert\,
  \P(\operatorname{Bin}(n, \hat{p}_D(a))\geq
  X_{c,t}(a))>\tfrac{\alpha}{2}\} / \hat  p_D(a)
\end{equation*}
and to the upper bound
\begin{equation*}
  \hat{Y}_{c,t-\tau}^{\text{upper}} = \sum_a \sup\{n\in\N \,\vert\, \P(\operatorname{Bin}(n, \hat{p}_D(a))\leq
  X_{c,t}(a))>\tfrac{\alpha}{2}\} / \hat  p_D(a),
\end{equation*}
where $\operatorname{Bin}(n, \hat{p}_D(a))$ is a binomial random
variable with parameters $n$ and $\hat{p}_D(a)$.

%\Rune{

\subsection{Unknown number of deaths per age group} \label{sec:known}
%
%For many countries, the age-specific fatalities $X_{c,t}(a)$ are unobserved. Under an additional assumption, 
%however, these can be estimated using the death rates from countries for which these data are available.  
Consider a country for which only the total number of fatalities $X_{c,t}$ are observed. 
By definition of $X_{c,t}$, we have that for every $a$,
\begin{equation*}
X_{c,t}(a) \given X_{c,t} \sim \text{Binom}(X_{c,t}, P_c(A = a \given D_{t} = 1)).
\end{equation*}
Given estimates $\hat P_c(A = a \given D_{t} = 1)$, we obtain
\begin{equation} \label{eq:xahat}
\hat{X}_{c,t}(a) = \hat \E[X_{c,t}(a) \given X_{c,t}] = X_{c,t} \cdot \hat P_c(A = a \given D_{t} = 1),
\end{equation}
%
which similarly to \eqref{eq:Y1} can be used to estimate the total number of infections $Y_{c,t-\tau}$. 
To estimate the probabilities $P_c(A = a \given D_{t} = 1)$, we require another invariance assumption. It states that 
the infection rates $a \mapsto P_c(I_{t} = 1 \given A = a)$ for different countries and different time points only 
differ by a multiplicative constant. 
%
\begin{ass}[Proportional infection rates] \label{ass:2}
For every fixed $a, \tilde{a}$, 
\begin{equation*} 
\frac{P_c(I_t = 1\,|\,A = a)}{P_c(I_t = 1\,|\,A = \tilde{a})} 
\end{equation*}
does not depend on $c$ nor $t$.
\end{ass}
%
%There are several equivalent formulations of the above assumption, see Section~\ref{sec:equiv}. 
Under Assumption~\ref{ass:2}, the probabilities $\P_c(A = a \given D_{t} = 1)$ can be estimated using
the death rates of another country,
see Proposition~\ref{prop:estimator} below. 
The result follows from two equivalent formulations of Assumption~\ref{ass:2}. These formulations
involve the death rathes, rather than the infection rates, and are therefore also more suitable for testing 
the validity of the assumption, see also Section~\ref{sec:uncert}. All proofs can be found in Appendix~A.
%We therefore state them explicitly below.

\begin{prop}[Equivalent formulation of Assumption~\ref{ass:2}] \label{prop:equiv1}
Let Assumption~\ref{ass:1} be satisfied. Then, Assumption~\ref{ass:2} is equivalent to the following statement. 
For every fixed $a$ and $\tilde{a}$, 
\begin{equation*} \label{eq:a2equiv2}
\frac{P_c(D_t = 1\,|\,A = a)}{P_c(D_t = 1\,|\,A = \tilde{a})}
\end{equation*}
does not depend on $c$ nor $t$.
\end{prop}
%
%As an immediate consequence, we obtain the following result.
%
\begin{prop}[Equivalent formulation of Assumption~\ref{ass:2}] \label{prop:equiv2}
Let Assumption~\ref{ass:1} be satisfied. Then, Assumption~\ref{ass:2} is also equivalent to the following statement. 
For all fixed $a$,  
\begin{equation} \label{eq:equiv2}
\frac{P_c(D_t = 1 \given A = a)}
{\sum_{\tilde{a}} P_c(D_t = 1 \given A = \tilde{a})}
\end{equation}
%
does not depend on $c$ nor $t$. 
\end{prop}
%
%
Using the above formulation, we obtain the following formula for calculating the probabilities 
$P_c(A=a \given D_t = 1)$ using the death rates from another country $c^\prime$.
%
\begin{prop} \label{prop:estimator}
Consider a fixed country $c$. Under Assumption~\ref{ass:1}~and~\ref{ass:2}, 
it holds that for all countries $c^\prime$, time points $t$ and age groups $a$, 
\begin{equation*}
P_{c}(A = a \given D_t = 1) =
\frac{
  \frac{
    P_{c^\prime}(D_t = 1\given A = a)}
    {\sum_{\tilde{a}} P_{c^\prime}(D_t = 1 \given A = \tilde{a})} \cdot P_{c}(A = a)}
  {\sum_{a^*}  
  \frac{
    P_{c^\prime}(D_t = 1\given A = a^*)}
    {\sum_{\tilde{a}} P_{c^\prime}(D_t = 1 \given A = \tilde{a})} \cdot P_{c}(A = a^*)}.
\end{equation*}
\end{prop}
%
We then compute the estimator \eqref{eq:xahat} by plugging in sample versions in the 
above expression. 
%In our current version of the app, we use data from Germany, China and 
%Italy to compute the above death rates. 
%}

%\Niklas{todo: Get bounds here.}


\section{Sources of uncertainty} \label{sec:uncert} We believe that
there are three main sources of uncertainty.

(i) Assumption~\ref{ass:1}: The case fatality rates are not known
exactly. Uncertainties in the estimated rates contribute to
uncertainties in the predicted values. Currently, we do not model this
type of uncertainty explicitly and rather use the estimated case
fatality rates from South Korea as if they were correct. However, in
the app this parameter can be changed manually, to get a feeling for
how it affects the estimated number of infected persons.

(ii) Assumption~\ref{ass:2}: We can check 
how well the data supports 
this assumption by using 
the number of COVID related deaths in each age group for different countries, 
and comparing the fractions \eqref{eq:equiv2} between each of these countries.
In particular, Assumption~\ref{ass:2} will not be satisfied exactly, but 
by considering several countries, we can 
infer the range of values to expect in \eqref{eq:equiv2}
and represent this as uncertainty in the predicted values. 
The current version of the app considers data that are currently available from a number of the disease epicenters.

(iii) Uncertainty from statistical inference: 
Suppose that in an age group, the case fatality rate is $0.1\%$ and we have 
$0$ fatalities in that age group.
A point estimate might then say that there are $0$ infected persons
in that age group. However, having $300$ infected persons, say,
is a reasonable explanation of the data, too (in that case, we would expect $0.001\cdot 300 = 0.3$ fatalities).
Currently, our estimate does not include this source of uncertainty.


\section{Mathematics and real life}
Assumption~\ref{ass:1} looks like an assumption about mathematics. 
But it is not. 
The conditional probability does not 
only describe how
the virus affects humans, 
but also how the health system treats the patients. 
There is a lot of staff working hard to 
keep this probability as small as possible. 
Thanks, all of you working in the health systems in all different countries, for 
your efforts to keep this number small. 




\section{Extrapolating infections into present/future}
We would have liked to work on this, but we were running out of time for this weekend, so we may come back to that question only later.

\section{Conclusions}

\begin{center}
\mybox{
\begin{minipage}{0.96\textwidth}
\centering
Our analysis suggests that the true number of infected people is a lot higher than the reported numbers -- please respect social distancing to avoid overburdening the health system. And stay safe!
\end{minipage}
}
\end{center}



\appendix

\section{Proofs}

\paragraph{Proof of Proposition~\ref{prop:equiv1}}
For every $a, \tilde{a}$ we have 
\begin{align*}
\frac{P_c(D_t = 1\,|\,A = a)}
{P_c(D_t = 1\,|\, A = \tilde{a})} 
&= 
\frac{
P_c(D_t = 1,A = a)
P_c(A = \tilde{a})}
{P_c(D_t = 1, A = \tilde{a})P_c(A = a)}  \\
&= 
\frac{
P_c(D_t = 1,I_t = 1, A = a)P_c(A = \tilde{a})}
{P_c(D_t = 1,I_t = 1, A = \tilde{a})P_c(A = a)} \\
&= 
\frac{
P_c(D_t = 1\,|\,I_t = 1, A = a)P_c(I_t = 1,A = a)P_c(A = \tilde{a})}
{P_c(D_t = 1\,|\,I_t = 1, A = \tilde{a})P_c(I_t = 1,A = \tilde{a})P_c(A = a)} \\
&= 
\frac{p_D(a)}{p_D(\tilde{a})}  \frac{P_c(I_t = 1\,|\,A = a)}{P_c(I_t = 1\,|\,A = \tilde{a})} ,
\end{align*}
and the result follows. $\hfill \square$

\paragraph{Proof of Proposition~\ref{prop:equiv2}}
For every $a$, we have that 
\begin{equation*} 
\frac{P_c(D_t = 1 \given A = a)}
{\sum_{\tilde{a}} P_c(D_t = 1 \given A = \tilde{a})}
= 
\frac{1}{\sum_{\tilde{a}} \frac{P_c(D_t = 1\,|\,A = \tilde{a})}{P_c(D_t = 1\,|\,A = a)}},
\end{equation*}
and the result now follows from Proposition~\ref{prop:equiv1}. $\hfill \square$


\paragraph{Proof of Proposition~\ref{prop:estimator}}
Using Proposition~\ref{prop:equiv2}, we have that for all $c^\prime$, $t$ and $a$, 
\begin{align*}
P_{c}(A = a \given D_t = 1) &= 
\frac{  
    P_{c}(D_t = 1, A = a)}
  {P_{c}(D_t = 1)}\\
    &= 
\frac{  
    P_{c}(D_t = 1\given A = a)
     \cdot P_{c}(A = a)}
  {\sum_{a^*}  
    P_{c}(D_t = 1\given A = a^*) \cdot P_{c_1}(A = a^*)
    }\\
&= 
\frac{
  \frac{
    P_{c}(D_t = 1\given A = a)}
    {\sum_{\tilde{a}} P_{c}(D_t = 1 \given A = \tilde{a})} \cdot P_{c}(A = a)}
  {\sum_{a^*}  
  \frac{
    P_{c}(D_t = 1\given A = a^*)}
    {\sum_{\tilde{a}} P_{c}(D_t = 1 \given A = \tilde{a})} \cdot P_{c}(A = a^*)} \\
&= 
\frac{
  \frac{
    P_{c^\prime}(D_t = 1\given A = a)}
    {\sum_{\tilde{a}} P_{c^\prime}(D_t = 1 \given A = \tilde{a})} \cdot P_{c}(A = a)}
  {\sum_{a^*}  
  \frac{
    P_{c^\prime}(D_t = 1\given A = a^*)}
    {\sum_{\tilde{a}} P_{c^\prime}(D_t = 1 \given A = \tilde{a})} \cdot P_{c}(A = a^*)},
\end{align*}
as desired. $\hfill \square$

\bibliography{ref}

\end{document}

%%% Local Variables:
%%% mode: latex
%%% TeX-master: t
%%% End:
